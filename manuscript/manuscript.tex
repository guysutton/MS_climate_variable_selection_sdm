% Options for packages loaded elsewhere
\PassOptionsToPackage{unicode}{hyperref}
\PassOptionsToPackage{hyphens}{url}
\PassOptionsToPackage{dvipsnames,svgnames,x11names}{xcolor}
%
\documentclass[
  authoryear,
  preprint,
  3p,
  twocolumn]{elsarticle}

\usepackage{amsmath,amssymb}
\usepackage{lmodern}
\usepackage{iftex}
\ifPDFTeX
  \usepackage[T1]{fontenc}
  \usepackage[utf8]{inputenc}
  \usepackage{textcomp} % provide euro and other symbols
\else % if luatex or xetex
  \usepackage{unicode-math}
  \defaultfontfeatures{Scale=MatchLowercase}
  \defaultfontfeatures[\rmfamily]{Ligatures=TeX,Scale=1}
\fi
% Use upquote if available, for straight quotes in verbatim environments
\IfFileExists{upquote.sty}{\usepackage{upquote}}{}
\IfFileExists{microtype.sty}{% use microtype if available
  \usepackage[]{microtype}
  \UseMicrotypeSet[protrusion]{basicmath} % disable protrusion for tt fonts
}{}
\makeatletter
\@ifundefined{KOMAClassName}{% if non-KOMA class
  \IfFileExists{parskip.sty}{%
    \usepackage{parskip}
  }{% else
    \setlength{\parindent}{0pt}
    \setlength{\parskip}{6pt plus 2pt minus 1pt}}
}{% if KOMA class
  \KOMAoptions{parskip=half}}
\makeatother
\usepackage{xcolor}
\setlength{\emergencystretch}{3em} % prevent overfull lines
\setcounter{secnumdepth}{5}
% Make \paragraph and \subparagraph free-standing
\ifx\paragraph\undefined\else
  \let\oldparagraph\paragraph
  \renewcommand{\paragraph}[1]{\oldparagraph{#1}\mbox{}}
\fi
\ifx\subparagraph\undefined\else
  \let\oldsubparagraph\subparagraph
  \renewcommand{\subparagraph}[1]{\oldsubparagraph{#1}\mbox{}}
\fi


\providecommand{\tightlist}{%
  \setlength{\itemsep}{0pt}\setlength{\parskip}{0pt}}\usepackage{longtable,booktabs,array}
\usepackage{calc} % for calculating minipage widths
% Correct order of tables after \paragraph or \subparagraph
\usepackage{etoolbox}
\makeatletter
\patchcmd\longtable{\par}{\if@noskipsec\mbox{}\fi\par}{}{}
\makeatother
% Allow footnotes in longtable head/foot
\IfFileExists{footnotehyper.sty}{\usepackage{footnotehyper}}{\usepackage{footnote}}
\makesavenoteenv{longtable}
\usepackage{graphicx}
\makeatletter
\def\maxwidth{\ifdim\Gin@nat@width>\linewidth\linewidth\else\Gin@nat@width\fi}
\def\maxheight{\ifdim\Gin@nat@height>\textheight\textheight\else\Gin@nat@height\fi}
\makeatother
% Scale images if necessary, so that they will not overflow the page
% margins by default, and it is still possible to overwrite the defaults
% using explicit options in \includegraphics[width, height, ...]{}
\setkeys{Gin}{width=\maxwidth,height=\maxheight,keepaspectratio}
% Set default figure placement to htbp
\makeatletter
\def\fps@figure{htbp}
\makeatother

\makeatletter
\makeatother
\makeatletter
\makeatother
\makeatletter
\@ifpackageloaded{caption}{}{\usepackage{caption}}
\AtBeginDocument{%
\ifdefined\contentsname
  \renewcommand*\contentsname{Table of contents}
\else
  \newcommand\contentsname{Table of contents}
\fi
\ifdefined\listfigurename
  \renewcommand*\listfigurename{List of Figures}
\else
  \newcommand\listfigurename{List of Figures}
\fi
\ifdefined\listtablename
  \renewcommand*\listtablename{List of Tables}
\else
  \newcommand\listtablename{List of Tables}
\fi
\ifdefined\figurename
  \renewcommand*\figurename{Figure}
\else
  \newcommand\figurename{Figure}
\fi
\ifdefined\tablename
  \renewcommand*\tablename{Table}
\else
  \newcommand\tablename{Table}
\fi
}
\@ifpackageloaded{float}{}{\usepackage{float}}
\floatstyle{ruled}
\@ifundefined{c@chapter}{\newfloat{codelisting}{h}{lop}}{\newfloat{codelisting}{h}{lop}[chapter]}
\floatname{codelisting}{Listing}
\newcommand*\listoflistings{\listof{codelisting}{List of Listings}}
\makeatother
\makeatletter
\@ifpackageloaded{caption}{}{\usepackage{caption}}
\@ifpackageloaded{subcaption}{}{\usepackage{subcaption}}
\makeatother
\makeatletter
\@ifpackageloaded{tcolorbox}{}{\usepackage[many]{tcolorbox}}
\makeatother
\makeatletter
\@ifundefined{shadecolor}{\definecolor{shadecolor}{rgb}{.97, .97, .97}}
\makeatother
\makeatletter
\makeatother
\usepackage{float}
\makeatletter
\let\oldlt\longtable
\let\endoldlt\endlongtable
\def\longtable{\@ifnextchar[\longtable@i \longtable@ii}
\def\longtable@i[#1]{\begin{figure}[H]
\onecolumn
\begin{minipage}{0.5\textwidth}
\oldlt[#1]
}
\def\longtable@ii{\begin{figure}[H]
\onecolumn
\begin{minipage}{0.5\textwidth}
\oldlt
}
\def\endlongtable{\endoldlt
\end{minipage}
\twocolumn
\end{figure}}
\makeatother
\journal{Ecological Modelling}
\ifLuaTeX
  \usepackage{selnolig}  % disable illegal ligatures
\fi
\usepackage[]{natbib}
\bibliographystyle{elsarticle-harv}
\IfFileExists{bookmark.sty}{\usepackage{bookmark}}{\usepackage{hyperref}}
\IfFileExists{xurl.sty}{\usepackage{xurl}}{} % add URL line breaks if available
\urlstyle{same} % disable monospaced font for URLs
\hypersetup{
  pdftitle={Climate predictor selection and variable reduction methods influence MaxEnt model performance and predictions of climatic suitability},
  pdfauthor={Clarke van Steenderen; Guy F. Sutton},
  pdfkeywords={keyword1, keyword2},
  colorlinks=true,
  linkcolor={blue},
  filecolor={Maroon},
  citecolor={Blue},
  urlcolor={Blue},
  pdfcreator={LaTeX via pandoc}}

\setlength{\parindent}{6pt}
\begin{document}

\begin{frontmatter}
\title{Climate predictor selection and variable reduction methods
influence MaxEnt model performance and predictions of climatic
suitability}
\author[1]{Clarke van Steenderen%
%
}
 \ead{vsteenderen@gmail.com} 
\author[1]{Guy F. Sutton%
\corref{cor1}%
}
 \ead{g.sutton@ru.ac.za} 

\affiliation[1]{organization={Center for Biological Control, Department
of Zoology and Entomology, Rhodes
University},city={Makhanda},postcode={6140},postcodesep={}}

\cortext[cor1]{Corresponding author}


        
\begin{abstract}
This is the abstract. Lorem ipsum dolor sit amet, consectetur adipiscing
elit. Vestibulum augue turpis, dictum non malesuada a, volutpat eget
velit. Nam placerat turpis purus, eu tristique ex tincidunt et. Mauris
sed augue eget turpis ultrices tincidunt. Sed et mi in leo porta
egestas. Aliquam non laoreet velit. Nunc quis ex vitae eros aliquet
auctor nec ac libero. Duis laoreet sapien eu mi luctus, in bibendum leo
molestie. Sed hendrerit diam diam, ac dapibus nisl volutpat vitae.
Aliquam bibendum varius libero, eu efficitur justo rutrum at. Sed at
tempus elit.
\end{abstract}



\begin{highlights}
\item Highlight 1\item Highlight 2\item Highlight 3
\end{highlights}


\begin{keyword}
    keyword1 \sep 
    keyword2
\end{keyword}
\end{frontmatter}
    \ifdefined\Shaded\renewenvironment{Shaded}{\begin{tcolorbox}[borderline west={3pt}{0pt}{shadecolor}, breakable, enhanced, interior hidden, sharp corners, frame hidden, boxrule=0pt]}{\end{tcolorbox}}\fi

\hypertarget{introduction}{%
\section{1. Introduction}\label{introduction}}

Ecological models are important tools to aid the development and
implementation of environmental policies and management programmes
\citep{Addison2013, Schuwirth2019, Sutton2022}. These models are used
for conservation planning \citep{Guisan2013}, predicting the
establishment and spread of invasive species \citep{Martin2020},
implementing biological control programmes
\citep{Sutton2019b, Mukherjee2021}, and forecasting species responses to
environmental change \citep{Bocedi2014}, amongst other applications.
Species distribution models (SDM's) are an example of ecological models
that have become increasingly popular in recent years \citep{Elith2009}.
SDM's typically take the form of correlative or mechanistic models that
correlate species presence/absences (or pseudo-absences) to
environmental covariates to identify suitable climatic conditions for
the study taxon \citep{Elith2011}. The Maximum Entropy species
distribution model (hereafter `MaxEnt') is amongst the most popular
methods for climate modelling studies and has been shown to perform well
compared to alternative modelling algorithms
\citep{Wisz2008, Phillips2017}. It uses maximum entropy to distinguish
between environmental conditions where the focal taxon is present from
environmental conditions at sites without confirmed presence records for
the taxon \citep{Elith2011}.

In recent years, a number of studies have investigated and demonstrated
that computational choices made during the model building process can
have a significant influence on resulting model outputs and inferences
drawn
\citep{Warren2011, Webber2011, Shcheglovitova2013, Boria2017, Sutton2022}.
Despite its importance in the model building process, covariate
selection methods have received considerably little attention to date
\citep[but see][]{Austin2011, Fourcade2018, Adde2023}, whereby covariate
selection refers to ``identify{[}ing{]} the best subset of covariates
out of a panel of many candidates, both from an ecological and
statistical perspective \citep[see][ and references therein]{Adde2023}.

\textbf{Next paragraphs}

\begin{itemize}
\item
  Review current methods used to select variables, e.g.~R2, VIF, PCA,
  expert opinion, and most recently, automated selection / model-based
  selection.
\item
  Aims of this paper are to show how variable selection can influence
  model outputs and inferences.
\end{itemize}

\hypertarget{methods-and-materials}{%
\section{2. Methods and materials}\label{methods-and-materials}}

\hypertarget{species-occurrence-records}{%
\subsubsection{2.1. Species occurrence
records}\label{species-occurrence-records}}

\hypertarget{environmental-predictors}{%
\subsubsection{2.2. Environmental
predictors}\label{environmental-predictors}}

Climate data were obtained by downloading the standard set of 19
bioclimatic variables from the WorldClim ver. 2.1 database (Fick and
Hijmans, 2017) (data available at: www.worldclim.org/download. html).
This dataset is representative of annual and seasonal means and
variation of temperature and precipitation metrics averaged over the
1950--2000 time period (current climate) at a 2.5 min resolution. These
variables have been shown to effectively model the climatic suitability
for non-native insects (e.g., Trethowan et al., 2011).

\hypertarget{model-calibration}{%
\subsubsection{2.3. Model calibration}\label{model-calibration}}

MaxEnt (ver. 3.4.3) was implemented in the ` dismo' R package (Hijmans
et al., 2021 ).

Given that MaxEnt is a presence/pseudo-absence modelling algorithm,
model calibration requires a user-defined geographic background to
sample the climate of representative grid cells where the focal species
is assumed to be absent (i.e., background points or pseudo-absences).
Background definition can have a significant effect on model output
(VanDerWal et al., 2009). The background should ideally represent the
geographic areas available to the focal species, omitting areas where
species absence is due to historical factors, dispersal constraints
and/or biotic interactions (Sanin and Anderson, 2018). Following Webber
et al.~(2011), we defined the model background using the Koppen-Geiger
climate classification system (available at: http://koeppen-geiger.vu-w
ien.ac.at). Only Koppen-Geiger climate zones that contained at least one
native-range occurrence record for D. rubiformis in Australia were used
as the background area from which background points were drawn for model
calibration (Fig. 1a). The Koppen-Geiger climate zones were intersected
with the occurrence records using the ` raster' R package (Hijmans,
2022). We randomly sampled 10 000 points (the default number used for
Maxent; Merow et al.~(2013) ) from within this background definition
using the `dismo' R package (Hijmans et al., 2021).

MaxEnt models were parameterised with default settings for multiple
parameters, including: convergence = 105, maximum number of iterations =
500 and prevalence = 0.5. The `fade by clamping' option was selected to
prevent extrapolation well outside the range of climatic values in the
model training area (Philips et al., 2017 ). Model predictions were
obtained using the `logistic output' to create continuous climatic
suitability raster layers scaled between 0 (climatically unsuitable) and
1 (climatically suitable).

\hypertarget{model-evaluation}{%
\subsubsection{2.4. Model evaluation}\label{model-evaluation}}

Model tuning experiments were applied to the native-range MaxEnt models
to derive within-sample evaluation metrics to guide the selection of
optimal MaxEnt parameter configurations (feature classes and
regularisation multipliers). Optimised parameter configurations would
then be used to refit the MaxEnt models before being projected into a
novel geographic region and making projections of climatic suitability
for D. rubiformis . Model tuning was performed by building MaxEnt models
with varying (1) feature class combinations (H = Hinge only, L = Linear
only, LQ = Linear and Quadratic and LQH = Linear, Quadratic and Hinge
features) and (2) regularisation multipliers (1:8). In total, 32 MaxEnt
models were specified. Native-range model performance and optimal
parameter configurations were assessed using 4-fold spatial block cross
validation using `ENMeval' (Kass et al., 2021).

Optimal parameter configurations were assessed using multiple metrics
that reflect different aspects of model performance. Four metrics were
calculated, including: (1) discriminatory ability (AUCtest), (2)
overfitting (AUCdiff), (3) omission rates (OR10), and (4) overall
parsimony (AICc). The use of AUC analyses for assessing the fit of
MaxEnt models has been criticised for a variety of reasons (see Lobo et
al., 2008; Peterson et al., 2008 ). However, AUC metrics are arguably
the most widely used metrics to evaluate MaxEnt model performance, and
as such, we believe it is important to include them in our evaluation,
and contrast the results obtained using AUC versus other metrics.

We specified five final MaxEnt models, four models calibrated with FC
and RM values that optimised model performance based on the
metricsdiscussed below, and a MaxEnt model calibrated with default FC
and RM values. Our intention was to compare MaxEnt model predictions and
perfor mance depending on which metric was used to select optimal
parameter configurations relative to the default MaxEnt settings.

\begin{enumerate}
\def\labelenumi{(\arabic{enumi})}
\item
  AUC test assesses the model 's ability to discriminate between
  predicted presence at withheld portions of the data used to test the
  model versus pseudo-absence points. An AUC of less than 0.8 is
  considered a poor model, between 0.8 and 0.9 is a fair model, between
  0.9 and 0.995 a good model, and \textgreater{} 0.995 an excellent
  model (Fielding and Bell, 1997). Thus, higher AUCtest values indicate
  increased ability to discriminate between testing and background
  points.
\item
  AUC diff is the difference between AUC values calculated on training
  points only (AUCtrain) and AUCtest {[}see (1) AUCtest above for
  details{]} (Warren and Seifert, 2011). Thus, higher AUCdiff values
  indicate whether the MaxEnt model is overfit on the training data, and
  thus, may perform poorly when evaluated against testing points.
\item
  OR 10 is the 10\% training omission rate (Boria et al., 2014 ).
  Overfit models have omission rates higher than the theoretical
  expectation for the threshold applied (Shcheglovitova and Anderson,
  2013). As such, the OR 10 criterion selected models calibrated with
  MaxEnt settings which best approximated the expected 0.10 omission
  rate. Models with omission rates increasingly higher than the expected
  value were considered to have a higher degree of overfit (Boria et
  al., 2017).
\item
  The Akaike Information Criterion corrected for small sample sizes
  (AICc) criterion simultaneously scores models according to their
  complexity and goodness-of-fit. AICc was used as the primary
  evaluation metric as it is calculated using MaxEnt models built using
  the entire species occurrence dataset (i.e.~all the occurrence points
  in the native-range), unlike AUC and OR10 (and numerous other metrics
  frequently used for model evaluation) which may be spatially biased
  due to the partitioning of the species occurrence dataset into
  training and evaluation sets (Sanin and Anderson, 2018). Optimal
  parameter configurations were determined by selecting model
  configurations which produced the lowest value for AICc (i.e., AICc=0;
  following Kass et al.~(2021)).
\end{enumerate}

\hypertarget{model-visualisation}{%
\subsubsection{2.5. Model visualisation}\label{model-visualisation}}

\begin{itemize}
\tightlist
\item
  Need to discuss how we map the different rasters, and maybe how we
  quantified the difference in suitability projections between climate
  layers?
\end{itemize}

All modelling and statistical analyses were conducted in R ver. 4.0.3 (R
Core Team, 2020). All values presented in text are presentedas mean ±
standard error, unless otherwise stated. A standardised ODMAP methods
protocol (Overview, Data, Model, Assessment, and Prediction) has been
completed for this study and can be found in Supplementary File S1.
ODMAP standardises the reporting of SDM modelling studies to improve
transparency and reproducibility (Zurell et al., 2020).

\hypertarget{results}{%
\section{3. Results}\label{results}}

\hypertarget{discussion}{%
\section{4. Discussion}\label{discussion}}

\hypertarget{declaration-of-competing-interest}{%
\section{Declaration of Competing
Interest}\label{declaration-of-competing-interest}}

The authors declare that they have no known competing financial
interests or personal relationships that could have appeared to
influence the work reported in this paper.

\hypertarget{data-availability}{%
\section{Data availability}\label{data-availability}}

All data and code required to reproduce the analyses are available in a
public GitHub repository:
https://github.com/guysutton/MS\_climate\_variable\_selection\_sdm

\hypertarget{acknlowedgements}{%
\section{Acknlowedgements}\label{acknlowedgements}}

CVS and GFS acknowledge funding from the South African Working for Water
(WfW) programme of the Department of Forestry, Fisheries and the
Environment: Natural Resource Management Programmes (DFFE: NRMP).
Funding was also provided by the South African Research Chairs
Initiative of the Department of Science and Technology and the National
Research Foundation (NRF) of South Africa. Any opinions, finding,
conclusions or recommendations expressed in this material are those of
the authors and the NRF does not accept any liability in this regard.

\hypertarget{supplementary-materials}{%
\section{Supplementary materials}\label{supplementary-materials}}


\renewcommand\refname{References}
  \bibliography{bibliography.bib}


\end{document}
